%%
%% Beuth Hochschule für Technik --  Thesis
%%
%% Appendix
%%
%%%%%%%%%%%%%%%%%%%%%%%%%%%%%%%%%%%%%%%%%%%%%%%%%%%%%%%%%%%%%%%%%%%%%


\chapter{Attached: The files of the package}


\subsection*{Stylefile}
The style file for this thesis is  \texttt{bhtThesis.sty}, which is available in the
archive file.  This must be \LaTeX\  must be findable, so it must be in a
\LaTeX\ known folder:
\begin{itemize}
\item Ubuntu-Linux: \verb|$HOME/texmf/tex/latex/bhtThesis/bhtThesis.sty|
\item MikTeX: \verb|c:\localtexmf\tex\latex\bhtThesis/bhtThesis.sty|
\end{itemize}

\subsection*{Sample Document}
This document is located in the subfolder  \texttt{tryout} of the zip-file. You can copy these files to a folder in which you will eventually work. The files are the following

\begin{itemize}
    \item \texttt{abstract\_en.tex} Abstract in English language
    \item \texttt{anhang.tex} the appendix
    \item \texttt{bhtThesis.bib} beinhaltet die zu zitierenden Literaturstellen und wird von bib\TeX ausgewertet 
    \item \texttt{main.pdf} is the output file with the print template
    \item \texttt{main.tex} contains the main document
    \item \texttt{makefile} realsiert das automatische mehrfache Übersetzen, hierfür muss \texttt{make} auf dem System installiert sein.
\item \texttt{myapalike.bst} beinhaltet die Formatierung für das
  Literaturverzeichnis 
\item \texttt{personalMacros.tex} kann einzelne, persönliche Macros beinhalten, die
  das Schreiben erleichtern
\item \texttt{titelseiten.tex} realisiert alle Seiten bis zum Beginn des ersten
  Abschnittes  

\item Ordner \texttt{pictures}
  \begin{itemize}
  \item \texttt{BHT-Logo-Basis.eps}
  \item \texttt{BHT-Logo-Basis.pdf}
  \end{itemize}

\item Folder \texttt{kapitel1}
  \begin{itemize}
  \item \texttt{ch1.tex} Source text of chapter 1
  \item Folder \texttt{pictures}
    \begin{itemize}
    \item \texttt{schaltbild.pdf}
    \end{itemize}
  \end{itemize}
  
\item Folder \texttt{kapitel2}
  \begin{itemize}
  \item \texttt{ch2.tex} Source text of chapter 2
  \item Folder \texttt{pictures}
    \begin{itemize}
    \item leer
    \end{itemize}
  \end{itemize}  
\end{itemize}

